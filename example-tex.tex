\documentclass{sensor-cdt-mres}
\usepackage{amsmath,amsthm}
\usepackage[backend=biber, citestyle=numeric-comp, style=nature]{biblatex}
\addbibresource{example-references.bib}


\title{Sensor CDT MRes Report}
\author{A.\,N. Author}
\college{Porterhouse College}
\date{24 August 2015}
\wordcount{2,500}



\begin{document}

\maketitle


\begin{abstract}
This document provides a brief overview of how to use the Sensor CDT MRes report \LaTeX template.
In addition to the document class, you will need the TeX Gyre Termes, Myriad Pro, XITS Math and DejaVu Sans Mono typefaces installed in your system.
It is important to define values for \texttt{\textbackslash title}, \texttt{\textbackslash author}, \texttt{\textbackslash college}, \texttt{\textbackslash date}, and \texttt{\textbackslash wordcount} in the document preamble.
\end{abstract}

\tableofcontents

\noindent Just start typing the main text after the abstract and table of contents.
Don't worry about setting any fonts of anything, this is done for you.

\LaTeX provides environments for producing lists.
We can create lists like
\begin{enumerate}
	\item the first item,
	\item the second item,
	\item the third item,
\end{enumerate}
by using the \texttt{enumerate} environment.
Similarly, we can create lists like
\begin{itemize}
	\item the first item,
	\item the second item,
	\item the third item,
\end{itemize}
by using the \texttt{itemize} environment.

After a while, we'll have finished the introduction and will want to move onto the next section.
This can be achieved by using the \texttt{\textbackslash section} command.

\section{A new section}
Notice that the section heading has been automatically added to the table of contents with the correct page number.
We can add subsections by using the \texttt{\textbackslash subsection} command.

\subsection{On mathematics and tables}
Again, the subsection heading (and page number) has been added to the table of contents automatically.

We can now write some mathematics, first as a display equation, using the \texttt{equation} environment:
\begin{equation}
	A = \frac{∂θ}{∂t} + u · ∇θ = 0.
\end{equation}
We can also write mathematics inline, like \( x^2 + y^2 = z^2 \).
If the equation should not be numbered, we should use the \texttt{equation*} environment.
This naming scheme applies to many environments, such that the version suffixed with an asterisk has no number.

Now we've moved onto the second column, without using any command.
The typesetting engine will decide when it's best to switch to the next column, as well as determining where tables and figures should sit.

As an example, we'll define a table just after this line in the source, although it won't necessarily appear just below in the typeset PDF.
\begin{table}[h]
	\centering
	\caption{One-column table of repeat length of longer allele by age of onset class.}
	\begin{tabular}{@{\vrule height 10.5pt depth4pt  width0pt}lccccc}
	& \multicolumn5c{Repeat length} \\
	\cline{2-6}
	Age of onset / years & \it n & Mean & SD & Range & Median \\
	\hline
	Juvenile, 2–20 & 40 & 60.15 & 9.32 & 43–86 & 60 \\
	Typical, 21–50 & 377 & 45.72 & 2.97 & 40–58 & 45 \\
	Late, > 50 & 26 & 41.85 & 1.56 & 40–45 & 42 \\
	\hline
	\label{tab:one-column-table}
	\end{tabular}
\end{table}
By giving the table a label, we can reference it automatically with \texttt{\textbackslash prettyref\{tab:one-column-table\}} to produce \prettyref{tab:one-column-table}.
For tables, the use of the \texttt{table*} environment means that the table spans two columns, but is still numbered.


\section{On figures}
We can define figures in a similar way, with the asterisked version spanning two column.
\begin{figure}
	\includegraphics[width=8.7cm]{example-figure}
	\caption{
		LKB1 phosphorylates Thr-172 of AMPKα \textit{in vitro} and activates its kinase activity.
	}
	\label{fig:one-column-figure}
\end{figure}
By giving the figure a label, we can reference it automatically with \texttt{\textbackslash prettyref\{fig:one-column-figure\}} to produce \prettyref{fig:one-column-figure}.

In addition, the \texttt{SCfigure} environment allows us to have a figure with the caption down the side.
This is useful for figures that are larger than one column wide, but not quite large enough to span the whole page.
\prettyref{fig:two-column-figure} demonstrates this.


\begin{SCfigure*}
	\includegraphics[width=11.4cm]{example-figure}
	\caption{
		LKB1 phosphorylates Thr-172 of AMPKα \textit{in vitro}
		and activates its kinase activity.
		This is a long caption that goes on and on to demonstrate how a figure with side caption would look.
		This kind of figure environment is useful when a figure is too wide for one column, but not wide enough to span an entire page.
	}
\label{fig:two-column-figure}
\end{SCfigure*}


\section{On references}
There are many ways of producing references in \LaTeX, with the two most popular being Bib\LaTeX and Bib\TeX.
These are outside the scope of this document, but in both cases citations are added with the \texttt{\textbackslash cite} command, which must be given the key of the reference you want to cite.
We can cite \cite{kadison1959}, \cite{anderson1981} and \cite{anderson1979} by using \texttt{\textbackslash cite\{kadison1959\}}, \texttt{\textbackslash cite\{anderson1981\}} and \texttt{\textbackslash cite\{anderson1979\}}.
The bibliography can then be added to the end of the document by using the \texttt{\textbackslash printbibliography}, for Bib\LaTeX, or \texttt{\textbackslash bibliography}, for Bib\TeX, commands.

\section{Appendices}
To add appendices, use the \texttt{\textbackslash appendix} command, followed by \texttt{\textbackslash section} commands for each appendix.


\printbibliography

\appendix

\section{Some supplemental material}
Note that this page does not appear in the page count in the declaration and only has one column.

\end{document}
